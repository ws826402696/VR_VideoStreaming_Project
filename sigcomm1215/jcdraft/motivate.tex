%!TEX root = main.tex

\section{Motivation}

We start with the background on \vr video streaming and the existing approaches.
We then outline the new \vr video-specific opportunities driven by a better understanding of how viewers perceive \vr video quality, and show the potential gains in user-perceived \vr video quality under limited available bandwidth. 
The section ends with a highlight on the key challenges to realize these potential gains.


\subsection{Background}

\mypara{\vr video streaming}
\vr videos have come to age.
By 2022, there will be over 55 million active VR headsets in the US, which is as many as paying Netflix members in the US in 2018~\cite{https://qz.com/1298512/vr-could-be-as-big-in-the-us-as-netflix-in-five-years-study-shows/}.
Almost all major content providers (YouTube~\cite{??}, Facebook~\cite{??}, Netflix~\cite{??}, Vimeo~\cite{??}, Amazon Prime~\cite{??}, Hulu~\cite{??}, iQIYI~\cite{??}, YouKu~\cite{??}) launched streaming services for \vr videos across many VR platforms~\cite{oculus,samsung,daydreams,etc}, believing that \vr videos are the future of story telling. 
Like non-\vr videos, 
A typical \vr video delivery pipeline is as following. 
A regular video is first encoded using a \vr encoder, and then just as regular videos, the \vr video will be chunked into segments, sent to a content delivery network (CDN) for Internet-scale distribution, and streamed from the CDN edge HTTP servers to \vr headset over the HTTP(S) protocol~\cite{hls,https://www.wowza.com/solutions/streaming-types/virtual-reality-and-360-degree-streaming}.


\mypara{Existing solutions}

\mypara{Bounded sensitivity to quality degradation}


\subsection{\vr video-specific opportunities}


\subsection{Potential improvement}


\subsection{Key challenges}




