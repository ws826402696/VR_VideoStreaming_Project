%!TEX root = main.tex

\section{Evaluation}
\label{sec:eval}

We evaluate \name with both trace-driven simulation and real user survey. 
Our key findingsare the following.

\begin{itemize}

\item Across a wide range of videos, \name achieves better QoE-bandwidth tradeoffs: \fillme\% better user-perceived QoE (in PSPNR) with the same bandwidth consumption, or \fillme\% less bandwidth consumption without drop in user-perceived QoE.

\item \name is more robust to bandwidth fluctuations and randomness of viewer actions than previous viewpoint driven baselines.

\item Both of the \vr video-specific tiling and quality adaptation contributed substantial improvement to the perceived QoE improvement.

\item \name imposes a very small amount of system overhead to both the client-side and server-side processing.

\end{itemize}

\subsection{Methodology}

\mypara{User survey}

\mypara{Trace-driven simulation}

\mypara{Synthetic viewpoint trace generation}


\subsection{End-to-end improvement}

\mypara{User survey experiment}

\mypara{Large-scale trace-driven simulation}


\subsection{Robustness}

\mypara{Impact of bandwidth fluctuation}

\mypara{Impact of noises in viewpoint trajectory}


\subsection{System overhead}

\mypara{Client-side overhead}
\jc{cpu, ram, bandwidth}

\mypara{Server-side overhead}


\subsection{Component-wise analysis}


