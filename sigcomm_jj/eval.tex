%!TEX root = main.tex

\section{Evaluation}
\label{sec:eval}

We evaluate \name with both trace-driven simulation and real user survey. 
Our key findings are the following.

\begin{itemize}

\item \name achieves higher video quality and uses less bandwidth than state-of-the-art baseline across a wide range of video genres: \fillme\% higher quality (in both user rating and PSPNR) with the same bandwidth consumption, or \fillme\% less bandwidth consumption without drop in PSPNR.

\item \name is robust to errors in the prediction of viewer actions and available bandwidth. 

\item The offline tiling and online adaptation contributed substantial improvement to the overall gain of \name.

\item \name imposes only negligible overhead and actually reduce the client-side computing overhead.

\end{itemize}

\subsection{Methodology}

\mypara{Survey-based evaluation}
\jc{please fill in, no more than 1 para}

\mypara{Trace-driven simulation}
\jc{please fill in, no more than 1 para}

\mypara{Synthetic viewpoint trace generation}
\jc{please fill in, no more than 1 para}


\subsection{End-to-end improvement}

\mypara{User survey experiment}

\mypara{Large-scale trace-driven simulation}


\subsection{Robustness}

\mypara{Impact of bandwidth fluctuation}

\mypara{Impact of noises in viewpoint trajectory}


\subsection{System overhead}

\mypara{Client-side overhead}
\jc{cpu, ram, bandwidth}

\mypara{Server-side overhead}


\subsection{Component-wise analysis}


